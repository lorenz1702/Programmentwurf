\chapter{Beschreibung der Funktionalität: Wer ist der Spion?}
Die entwickelte App ist eine digitale Umsetzung des beliebten Gesellschaftsspiels "Wer ist der Spion?". Ziel der App ist es, den Spaß und die Spannung des Spiels auf Mobilgeräte zu bringen und es den Benutzern zu ermöglichen, jederzeit und überall zu spielen. Die Hauptfunktionen der App umfassen:

\begin{itemize}
    \item \textbf{Spielrunden erstellen:} Benutzer können neue Spielrunden erstellen und Mitspieler einladen, entweder aus ihrer Kontaktliste oder durch Generierung eines Einladungslinks.
    \item \textbf{Rollen zuweisen:} Die App weist automatisch jedem Spieler eine Rolle zu, entweder als normaler Spieler mit Kenntnis des geheimen Themas oder als Spion, der das Thema nicht kennt.
    \item \textbf{Fragen stellen und Antworten geben:} Die Spieler können sich gegenseitig Fragen stellen, um Hinweise auf das geheime Thema zu erhalten, während der Spion versucht, das Thema zu erraten, ohne entdeckt zu werden.
    \item \textbf{Punkte vergeben und Spielstatistiken anzeigen:} Die App verfolgt die Spielrunden und Punkteverteilung, um den Spielfortschritt zu verfolgen, und zeigt den Spielern ihre Statistiken an.
\end{itemize}

\section{Beschreibung des Kundennutzens:}

Die App bietet den Benutzern eine unterhaltsame und interaktive Möglichkeit, das Spiel "Wer ist der Spion?" zu spielen, ohne physische Karten oder Zubehör zu benötigen. Der Kundennutzen umfasst:

\begin{itemize}
    \item \textbf{Flexibilität:} Spieler können das Spiel jederzeit und überall spielen, indem sie einfach ihre Mobilgeräte verwenden.
    \item \textbf{Soziale Interaktion:} Das Spiel fördert die soziale Interaktion zwischen den Spielern, da sie sich gegenseitig Fragen stellen und versuchen, den Spion zu entlarven.
    \item \textbf{Spaß und Spannung:} Die App bietet den Spielern die Möglichkeit, den Nervenkitzel des Spiels zu erleben, während sie versuchen, das geheime Thema zu erraten oder ihre wahre Identität als Spion zu verbergen.
\end{itemize}



\subsection{Verwendete Technologie}
Die Anwendung wird in Kotlin entwickelt, einer modernen und leistungsstarken Programmiersprache, die sich besonders gut für die Entwicklung von Android-Apps eignet. Weitere Technologien und Frameworks umfassen:
\begin{itemize}
    \item \textbf{Back-End:} Kotlin 
    \item \textbf{Front-End-Framework:} Kotlin-Ktor Client
    \item \textbf{Entwicklungsumgebung:}  IDE oder Code-Editor
\end{itemize}


\subsection{Ausführbare Datei}

Man kann das Projekt als Jar kompilieren und ausführen lassen. Unter diesem \href{https://github.com/lorenz1702/Spy-Game/commit/ea4fd2fed71a7a43e10448dd173c79ca53789201}{Commit-Hash} unter build/libs/SpyGame.jar
